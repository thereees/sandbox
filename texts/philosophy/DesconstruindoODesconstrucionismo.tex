\documentclass[10pt,a4paper]{book}
\usepackage[utf8]{inputenc}
\usepackage{fontspec}
\usepackage{amsmath}
\usepackage{amsfonts}
\usepackage{amssymb}
\usepackage{graphicx}
\usepackage{hyperref}
\usepackage{xltxtra} 
\usepackage{xgreek}  
\usepackage{polyglossia}
\setdefaultlanguage{Portuguese}
\title{Desconstruindo o Descontrucionismo}
\author{Pierre Magnard}
\date{10 de julho de 2017}
\begin{document}
	\maketitle
	
	\chapter{Desconstruindo o Descontrucionismo}	
	\section{Video}
	
	\url{https://youtu.be/kJ1AntkVcq4}
	
	Tradução:\\
	Fabrício Soares de Souza\\
	\\
	Pierre Yves:\\
	Professor Pierre Magnard veio compartilhar conosco esta apresentação sobre como desconstruir a desconstrução.\\
	\\
	Pierre Magnard:\\
	Muito obrigado. É realmente uma imensa honra e uma grande alegria recebê-lo.\\
	\\
	Pierre Yves:\\
	Muito obrigado, professor.\\
	\\
	Pierre Magnard: \\
	Sou eu quem agradeço Pierre Yves a este  convite tão amável. Eu não poderia faltar com o seu chamado, mesmo sendo às pressas. E eu me comprometo de fazer então, com muito prazer, uma improvisação que não é uma qualquer pois ela é mais do que isso: uma confidência.\\
	\\
	A luta contra os "desconstrutores" eu fiz durante toda minha vida universitária. Desde a minha chegada ao Liceu Henri IV em 1947 até os dias de hoje. E a luta não acabou. Porque podemos colocar na conta desses senhores uma palavra esplêndida, uma palavra que nos faria acreditar na inteligência de seu autor. Se ela não foi pronunciada por desprezo ou por acaso.\\
	\\
	"Em uma multidão, há aqueles que têm êxito e há aqueles que não são nada."\\
	\\
	Eu acolhi isto com uma alegria no coração. Eu me reconheci no grupo daqueles que não são nada. \\
	E que nunca quiseram ser outra coisa do que isso. Nada, é no nada que Deus se esconde.\\
	\\
	Meu antigo aluno e amigo, Christian Bobin, o repete ao longo de páginas:\\
	\\
	"Deus se esconde no nada".\\
	\\
	E, nós mesmos, se somos capazes de reencontrar a virtude essencial que fez o Homem, isto é, a humildade. Se nós somos capazes de partir do personagem vilão que a sociedade gostaria de nos fazer representar. Nós somos, exatamente, o que Homero chamou oυδείς (U'dis), isto é, ninguém. Nem mesmo um oυδείς (U'dis), mas ninguém, ou seja, nada. É este nada que transcende todos os personagens e que nos dá a capacidade de encontrar um outro na amizade e no amor e, também, de aspirar encontrar Deus.\\
	\\
	Mas, a título de preâmbulo, tentemos ver como eu me deparei com a desconstrução\\
	\\
	Eu iniciarei de um texto que mereceria ser recolocado à moda, hoje. É o romance "A Náusea" de Sartre de 1937. O Liceu do Havre, como por acaso, é um lugar onde o gênio maléfico passeia. No Liceu do Havre, há um grande pátio no qual nos deparamos com um imponente castanheiro. E alguns de vocês que já leram esse livro de Sartre se lembram da parte surpreendente sobre a raiz do castanheiro que desperta em Sartre uma sensação de nojo a tal ponto que ele tem vontade de vomitar diante do que ele chama: "o surgimento de um ser que é sempre demasiado".\\
	\\
	Eis aí, no fundo, a primeira manifestação do espírito moderno.\\
	\\
	Pois onde eu encontro a náusea, a tradição metafísica, desde a Grécia antiga, falava de "espanto/perplexidade", θαυμαζείν (thaumazein) de maravilhamento, de contentamento. Alegria diante do mistério da aparição do Ser. O Ser dava-lhe contentamento.\\
	\\
	Ora, a grande novidade do homem moderno é que o Ser lhe causa náusea. E o Ser, como um acréscimo, é vivido como algo demasiado. É a palavra que utilizará Sartre em "O Ser e o Nada" quando ele retoma esta memorável análise:\\
	\\
	"O Ser é sempre demasiado."\\
	\\
	E é a partir daí que começa esse caminho da destruição ou desconstrução que almejará gerir, nada menos que, a metafísica enterrando a metafísica.\\
	\\
	Quando eu cheguei ao Liceu Henri IV para me preparar à École Normale Supérieur. O que eu encontrei? Eu encontrei, precisamente, o grande debate aberto, no qual se podia ser metafísico sem ser absolutamente medíocre, ultrapassado. E pensar, hoje em dia, não exige essa renúncia à metafísica.  Portanto, vou tentar dizer-lhes, de modo bem simples, o que isto pode significar pois vocês não são, provavelmente, metafísicos experientes. Do que se trata?\\
	\\
	Eu diria que há duas palavras que a tradição metafísica, desde os gregos, associa, a saber: Deus e o Ser. Duas palavras entre as quais parece se estabelecer uma nuvem sem que sejamos constrangidos de identificar Deus ao Ser e fazer do Ser a expressão mesma de Deus. Mas temos esta noção de Ser na referência a Deus e na nominação mesma de Deus.\\
	\\
	Vocês sabem como um grande filósofo francês, Étienne Gilson, vai sinalizar no desenvolvimento do que ele chamará de metafísica do Êxodo fundada a partir do texto do Êxodo 3,14 no qual Deus guardando seu nome responde a Moisés:\\
	\\
	"Eu sou o que sou"\\
	\\
	Eu estou de minha parte, dado que sou uma cria da desconstrução, pelo fato da minha chegada em 1947 ao Liceu Henri IV, reservado no lugar do gilsonismo que fora minha filosofia provisória até então. Reservado no sentido que eu não estou seguro que Deus se nomeia (revela) na sarça ardente e que o Ser seja a designação mesma de Deus. Mas, seguindo o caminho de Gilson, o que vocês podem compreender bem, é que ele vai da afirmação de Deus saber qual é a relação entre Deus e o Ser.\\
	\\
	Qual é a relação entre Deus e o Ser?\\
	\\
	Vamos então para 1947. Isto não nos diz nada? A maioria de vocês não tinha nascido. Eu posso lhes dizer que foi um período sombrio. Um período inquietante. Um período em que os franceses não eram franceses. Um período em que se podia denunciar por nada e arrastar para o tribunal qualquer um por um dizer, por pouca coisa. 1947 é a época da corte marcial. 1947 é a época da depuração.\\
	\\
	Logo, imaginem vocês como era. Um período, portanto, em que falar poderia ter sérias consequências. E era preciso saber o que se poderia avançar. Deus, o Ser mas, afinal, o Ser o que é?\\
	\\
	Se se admite que é o conceito mais vasto. O mais compreensível. O mais capaz de abraçar a realidade e de captá-la em seu sentido. O Ser seria ser, talvez, a operatividade divina. Se não for o próprio Deus. Seria, talvez, a criação. Seria, talvez, o Reino de Deus. Seria, talvez, a energia divina. Ou como dizem os ortodoxos: as energias divinas. Eu creio que ninguém recusará esta asserção.\\
	\\
	Assim mesmo, não se faria do Ser um nome de Deus mas seríamos levados a estabelecer uma relação entre o Deus onipotente e o Ser. O Ser sendo, exatamente, a expressão da operatividade criadora. Ora, o que se professa naqueles anos de 1947, 1948, 1949? Professa-se, precisamente, a inconsistência de um mundo que se fragmenta. E a ausência total de significação à esse destino do mundo. Dito de outro modo, vive-se uma crise do sentido. Vive-se uma crise da significação.\\
	\\
	Como repete meu amigo George Steiner: "é o sentido do sentido que parece se subtrair" a partir dessa época o sentido do sentido, o sentido de que se nutre.\\
	\\
	Mais perto de nós, um dos meus amigos jovens, Richard Millet, o qual vocês conheçam, talvez, os panfletos completamente formidáveis, fala de uma exaustão do sentido para caracterizar a nova  mentalidade. Entramos em uma era de exaustão do sentido. O que quer dizer por exaustão do sentido? Richard Millet tem uma expressão absolutamente correta:\\
	\\
	"Vivemos a exaustão do sentido até experimentarmos a vertigem do nada."\\
	\\
	O que acontece é que naquele momento há uma espécie de má consciência de toda a intelligentsia francesa para com o seu passado, para com sua história vivida. Gosto de citar o poeta René Char que retomará nas páginas de Hypnos de 1948, num discurso que ele proferiu em 1943:\\
	\\
	"Nossa herança não nos foi dada por nenhum testamento."\\
	\\
	E ele vai tentar analisar esta negação geral dos franceses para com sua cultura. Ele vai tentar analisar esse desleixo geral para com todas as tradições por essa ausência de testamento. E se  compreendemos, René Char nos diz em 1943: "Nossa herança não nos foi dada por nenhum testamento." Em outras palavras, nós somos herdeiros encarregados de  guardar um patrimônio intelectual e espiritual. Este patrimônio, de repente, é atingido por um desinteresse. Ele parece caduco. Como se a nova geração não quisesse assumi-lo.\\
	\\
	De quem se trata? De René Bazin, Henry Bordeaux, Paul Bourget\\
	Ainda de quem se trata? De Charles Maurras\\
	De quem se trata? uma vez mais: De Maurice Barrès\\
	\\
	Resumindo, trata-se de toda uma literatura que se dizia francesa e que nos inscrevia em nossas raízes, em nosso passado. Ora, voltem às revistas dos anos de 1946, 47, 48 e vejam como são tratados estes autores. Há uma espécie de negação geral de reconhecer os mesmos. Até do gentil Jean Giraudoux que é condenado por causa da sua situação complexa e seu papel controverso frente à ocupação alemã. Eles são todos estigmatizados porque eles não teriam compreendido ou porque não teriam tido consciência da reviravolta dos anos sombrios. E eles não teriam percebido os novos ventos da História.\\
	\\
	Portanto, tentar compreender a situação da inteligência francesa naqueles anos é que, de algum modo, era indecente reivindicar o nobre passado da França, sua precedência, a fortiori, de se nutrir.\\
	\\
	Deve-se rejeitar tudo isto.\\
	\\
	E aparecem novos autores que ganharão reputação durante esse período de negação. Se penso num destes que conheci, citarei - que eu encontrei - François Mauriac. \\
	\\
	Sobre o caminho de Chartres. Eu era estudante. Acabamos de nos interpelar, com consciência pesada,\\
	explicando que nossa vontade de ser fiel a nossa tradição é uma recusa de humanidade e que deveríamos 'cortar o nosso pescoço'. Eis o que nos ensinavam. Recusa de humanidade porque gostaríamos de ser fiéis a nossos valores franceses.\\
	\\
	Além de Mauriac, eu citarei André Malraux. Foi um pouco mais tarde quando eu o encontrei em Mendig(?) para uma apresentação dele que já era um homem célebre. Então, eu entro na pequena cabine na qual ele se preparava para sua intervenção, forço a porta e digo-lhe: "mestre, diga-nos, o que devemos entender por Gaullismo?"\\
	\\
	Ele disse estas palavras sublimes: "O Gaullismo é a fidelidade à oportunidade de fidelidade." Foi André Malraux que me disse, pessoalmente.\\
	\\
	Heidegger\\
	\\
	A partir disso, então, vocês podem imaginar como era aquele momento bem difícil dos anos que vieram após a guerra. Foi o momento, no qual precisamente, vai aparecer a tentação da desconstrução. Então, como a tentação da desconstrução se manifesta?\\
	\\
	Na khâgne do liceu Henri-IV, isto é, classe preparatória para os cursos superiores (universitários) nós tínhamos como professor de filosofia, Jean Beaufret. Jean Beaufret era aluno de Martin Heidegger. E Jean Beaufret ia todo final de semana levar a Heidegger as questões que lhe colocavam os alunos do curso preparatório do liceu Henri-IV.\\
	\\
	Era eu, portanto, que, às terças-feiras de manhã, tinha as respostas diretas de Heidegger às nossas perguntas. Eu, portanto, garanti a conversação entre os estudantes de Paris e Heidegger durante dois anos. E tenho bastante orgulho de ter orientado o debate já que era eu quem fazia as perguntas. Portanto, isso, posso talvez, enfim, dizer-lhes, pois eu nunca relatei a ninguém durante toda minha vida. Mas tendo a necessidade de dizê-lo no período, digamos, particularmente desastroso que vivemos hoje em dia.\\
	\\
	Eu tive a honra de, durante dois anos, ser o responsável por transcrever as aulas de Jean Beaufret para juntar todas as perguntas dos meus colegas e daí colocá-las a Heidegger. Então, que questões colocávamos a Heidegger? Essencialmente, queríamos saber o que era o Ser? E Heidegger nos respondia pelo seu livro de 1927: \\
	\\
	"o Ser deve ser interpretado segundo todas as modalidades do tempo." Ser e Tempo (Sein und Zeit).\\
	\\
	Em outras palavras, o Ser não era mais a denominação do estável, do imutável, do eterno, do absoluto como dizia Étienne Gilson que eu citei anteriormente. Ele não era mais, portanto, a denominação do próprio Deus. O Ser era a conotação de um evento, de um surgimento na História da própria divindade. Por que não? Pois, de fato, Heidegger que tinha abandonado seu catolicismo de origem para aderir ao protestantismo de sua esposa, via, essencialmente, Deus em sua manifestação no tempo. O tempo é o acontecimento em que se dá, em que se desvela o divino. Dito de outra maneira, se eu digo "é", "foi", "será" estas são as conotações de um evento, de uma aparição, de uma produção do divino na história. E, assim, se se considerava Deus como determinação do Absoluto, do incondicionado, do Princípio passa-se a considerar o divino em sua eventualidade (como um acontecimento, como aquilo que se mostra no presentificar, em seu desvelamento).\\
	\\
	Isto é algo que sempre me deu muito trabalho: ver como Heidegger mudou a visão ainda tomista da teologia, que a classe preparatória do liceu Henri-IV tinha, para uma visão barthiana (de Karl Barth) da teologia. Isto é, para uma concepção protestante.\\
	\\
	Era a tentação de todos os meus colegas e, também, minha, evidentemente. E nós dávamos uma grande atenção àquela proposta de Heidegger. Esta preocupação me levou a estudar durante toda minha vida os textos de Heidegger sobre o Ser. E encontrei num número bem grande desses textos que todos insistem sobre o fato de que o Ser não é, necessariamente, uma designação de Deus porém, não obstante, seja a afirmação de Deus na história. Eu cito o próprio Heidegger:\\
	\\
	"Ser e Deus não são idênticos e nunca conceberei pensar o sentido de Deus através do Ser. Alguns de vocês talvez saibam que eu venho da teologia. Assim, eu guardei um amor antigo por ela. E compreendo alguma coisa. Se eu escrevesse ainda uma obra de teologia, foi porque fui tentado de fazê-la. Por isso, a palavra Ser não deveria intervir. Ao mesmo tempo, o pensamento do Ser não é necessário. Quando se tem necessidade dele, já não é mais a fé. Ou seja, quando se tem a necessidade do Ser já não é mais fé. É o que Lutero compreendeu. Mesmo em sua própria igreja parece-se esquecer. Meu pensamento é muito modesto para aqueles que têm a aptidão de pensar com o Ser o sentido de Deus de maneira teológica. Do Ser nós não podemos obter nada para esta ciência. Eu creio que o Ser jamais poderá ser pensando como fundamento e sentido de Deus. Mas que, todavia, a experiência de Deus se fazer uma revelação possível. Advém, em sentido próprio, da dimensão do Ser. O que não significa, entretanto, que o Ser possa valer como predicado possível de Deus."\\
	\\
	Foi para mim que ele endereçou este texto. Já que ele era no início seminarista no seminário católico antes de seu esgotamento rumo ao luteranismo. "Ao mesmo tempo, o pensamento do Ser não é necessário". Mas o que significa "ao mesmo tempo" para Heidegger? "Ao mesmo tempo" é uma maneira de habitar o incognoscível. Não é nada além disso. Pois uma vez que ele se constitui sem a razão. Esta não é a fé de Joseph Ratzinger.\\
	\\
	Portanto, o Ser não é um predicado de Deus. O Ser não designa Deus. Mas ele vai fazer uso da palavra Ser no ato de dizer o acontecimento da revelação. Isto é exatamente o protestantismo barthiano. Ora, é isto que vai estar na origem da famosa desconstrução. Por quê e como?\\
	\\
	Bem, o que é preciso é que eu utilize com vocês a pedagogia de Martin Heidegger para nossa própria situação. Heidegger, como procuramos explicar a partir do que significou a remoção da noção do Ser, que fora excluído do discurso teológico, por sua iniciativa - nos dizia isso, ao nos reenviar ao capítulo 6 de Ser e Tempo (Sein und Zeit) de 1927. E nesse capítulo 6, está a questão, no fundo, da destruição ou como traduzirá Gérard de Granel: desconstrução.\\
	\\
	O texto mereceria ser aqui analisado, mas seria necessário toda a noite para fazê-lo. Vou dizer, simplesmente, o que ele contém. Heidegger nos diz:\\
	\\
	"Eu proponho uma desconstrução da linguagem comum e do pensamento usual. O pensamento usual desde Aristóteles conjuga Ser e Deus. A oposição de identidade destes dois conceitos, eu demonstro que se pratique uma disjunção entre Deus e Ser. E esta disjunção eu a chamo: desconstrução. Não pratiquemos nenhuma desconstrução mais pois todos os outros binômios da linguagem comum antiga são binômios que portam uma sabedoria que nós devemos alcançar. Todo o pensamento estruturado de antônimos, sinônimos, identidades e suas identidades como suas oposições,  [devem ser] conservados. A desconstrução deve, portanto, ser concebida de uma maneira crítica."\\
	\\
	Como dirá Léon Brunschvicg:\\
	\\
	"Herança de Palavras; Herança de Ideias" \\
	\\
	Em outras palavras, vocês sabem que em todos esses binômios têm uma sabedoria que não devemos renunciar. "A desconstrução deve, portanto, ser concebida de uma maneira crítica." É a palavra que Heidegger utilizava. E fazer a crítica da desconstrução quer dizer em qual circunstância ela é permitida. E ela é permitida, unicamente, para efetuar a disjunção entre Deus e o Ser.\\
	\\
	É algo de muito significativo esta insistência.\\
	\\
	Então, prosseguindo, para manter certa reserva e cuidado, principalmente face a alguns outros desconstrutores que utilizaram Heidegger de maneira indecente, insolente e, completamente, inaceitável. Até nos fazer 'colocar nele um uniforme da SS' e tratá-lo como um nazista. Passemos por cima dessa calúnia absolutamente gratuita.\\
	\\
	Eu me permiti de ter relações pessoais com o padre abade de Saint Martin de Beuron. A abadia que fica a 4 quilômetros de Messkirch, a cidade natal de Heidegger. A abadia aonde Heidegger, acompanhado dos pais, ia, todos os domingos, à missa no tempo em que ele era ainda fiel ao catolicismo. A abadia onde ele também fez conferências magistrais em 1931, 1932 e mesmo em 1933 até o dia em que ele mudou de ideia e se tornou reitor de sua universidade (Universidade de Friburgo).\\
	\\
	Ele era talvez...ele não merecesse ser persona grata em uma abadia beneditina. Estávamos em todos os debates mas ele não estava mais habilitado a dar conferências magistrais. "Heidegger", disse-me o atual padre abade, "permanecerá fiel a abadia de Beuron até o fim de seus dias. Mas ele não [vai] mais à missa na parte da manhã. Ele a frequentava somente ao anoitecer. No serviço litúrgico vespertino do conflito." No qual ele declarará, numa carta que ficou famosa, que esse serviço de Liturgia das horas da Véspera, no limiar da noite, é o serviço litúrgico no qual o mundo morre e no qual ele via, à sua maneira, a morte de Deus.\\
	\\
	Eis, portanto, um tipo de Heidegger romântico que guarda da fé cristã uma certa nostalgia. Mas que, no entanto, vai do desespero à desconstrução. \\
	\\
	Dito isto, o que será a desconstrução? E como ela vai se desenvolver?\\
	\\
	Bem, eu diria que a desconstrução vai ser realizada dia a dia na universidade da minha geração. O que quer dizer que eu recebi o forte choque desde seu início e que foi muito difícil aguentá-lo. Quem foram os quatro cavaleiros negros do apocalipse do mal?\\
	\\
	O primeiro, eu diria que foi Gilles Deleuze; o segundo, Michel Foucault; o terceiro, Pierre Bourdieu e o quarto, Jacques Derrida.\\
	\\
	Eu pego somente estes quatro simplesmente porque eu tive a honra de ser o desafiante deles, de ser o antípoda dos mesmos, de ser o adversário deles. E de ter sido em toda a minha vida o fraternal adversário dos quatro. Amaldiçoando suas doutrinas enquanto procurava conservar um mínimo de relação amigável com eles. De tal modo que seria, talvez, interessante que um dia eu revelasse minhas lembranças desses quatro cavaleiros negros.\\
	\\
	O que eles representam? Comecemos pelo primeiro:\\
	\\
	Gilles Deleuze\\
	\\
	Eu o conheci melhor. Ele era mais velho do que eu dois anos. Ele entrara na École Normale Supérieur quatro anos antes de mim porque eu era mais meticuloso e ele era um fardón (fanfarrão, exibido), precoce. Mas lá na École Normale Supérieur nós criamos uma espécie de confraria que nos tornou solidários uns aos outros. Para o melhor e para o pior. Portanto, eu frequentei bastante Gilles Deleuze e tentei compreender o que nos levou ao seminário Alquié (Ferdinand) no qual ele era o protagonista.\\
	\\
	Então, o intuito de Deleuze de retomar a operação de desconstrução e de prolongá-la foi muito rápido. Prolongar a desconstrução, quer dizer, efetuá-la sobre todos os binômios reputados sacrossantos e que não mereceriam talvez essa reputação.\\
	\\
	Eu me explico: o que significa fazer a desconstrução?\\
	\\
	Se se considera a vulgata filosófica, a filosofia tradicional, ela é fundada naquilo que se chama convertibilidade do sentido da expressão onto (Ser). Ou seja, é de bom tom dizer que o Bem, o Belo e o Verdadeiro são convertíveis. Portanto, não se procurará a Beleza no mal. As flores do mal não funcionam. Não se procurará a Beleza no sacrilégio. Não se procurará a Beleza na profanação como faz a arte contemporânea. Já que a Beleza é o esplendor do Bem e, ao mesmo tempo, é o esplendor da Verdade. Quer dizer que não se procurará a Beleza na mentira. Em suma, há aí identidades notáveis, como se diz na filosofia, que visam que o Bem seja o Belo e que o Belo seja o Verdadeiro.\\
	\\
	Ora, a primeira tese de Gilles Deleuze busca refutar estas identidades notáveis. E de declarar que seria, talvez, inteiramente interessante de procurar a Beleza na linhagem (?), na perversão... procurar a Beleza no mal. Vejam como é feita a inversão e como a penetração, com força de uma literatura de uma ética completamente perversa, se efetua em favor da rejeição de identidades notáveis.\\
	\\
	Eis aí a desconstrução em curso.\\
	\\
	E essa desconstrução vai nos conduzir a um reexame de tudo o que a linguagem nos impõe. Gilles Deleuze era muito sedutor e inteligente, é preciso dizer, quando ele tentava nos convencer que nosso modo de pensar era, essencialmente, condicionado por estruturas...  estruturas gramaticais que impunham que o sujeito fosse o pivô da proposição e, de acordo com entendimento dele, os epítetos ou os atributos sejam, em conjunto, conjugados e distribuídos.\\
	\\
	Ora, isto tratará de tentar subverter, perverter a ordem gramatical levando-nos a imaginar um tipo de frase cujo sujeito não seria mais o princípio ao propor que todos os atributos podem  tornar-se sujeitos. E isto Deleuze toma, incansavelmente, como a imagem da árvore. A árvore é um paradigma falso pois ele nos impõe um princípio e consequências. Ele nos impõe um ponto de partida e um ponto de chegada. Ele nos impõe uma genealogia de nosso pensamento como também de nossas vidas e nossas ações. Ele nos impõe a consideração de uma paternidade que o princípio simboliza.\\
	\\
	Ora, ele concebe, a partir daí, uma imagem de um mundo que não seria feito de árvores mas de rizomas. O rizoma vocês sabem o que é? É a batata. É o efeito de um desenvolvimento sem raízes por radículas que partem do tubérculo e que permitem um outro tubérculo nascer, e assim por diante, sem que haja um tronco.\\
	\\
	Ora, para Deleuze, é preciso renunciar a imagem do tronco e raízes e pensar segundo a imagem das radículas. Portanto, a imagem do rizoma (rizot?) que faz tábula rasa de tudo o que é raiz e de tudo que é enraizamento. Deleuze vai recrudescer nesse tipo de crítica. Ele vai recrudescer porque ele começa aplicá-la à nossa sociedade.\\
	\\
	Há uma sociedade na qual nós vivemos que é uma sociedade fundada a partir da família. E ele defenderá que a sociedade seja calcada sobre a multidão, o bando. Sobre a multidão e não mais sobre a família.\\
	\\
	Separo algumas páginas absolutamente impressionantes, nesse livro que venho lhes mostrar - Mil Platôs de Gilles Deleuze -, com a apologia da multidão.\\
	\\
	"É preciso que a humanidade não seja fundada sobre a família mas sobre a multidão."\\
	\\
	E para isto que o leva à entender, mais uma vez, sua desconstrução e querer imaginar que a crítica que se efetua ao nível da sociedade seja efetuada ao nível do próprio ser vivo. Daí sua noção de um corpo sem órgãos. É preciso renunciar a organicidade. Deve-se considerar que em um corpo qualquer órgão pode fazer qualquer coisa... em uma vicariância geral. \\
	\\
	Mas a organicidade foi proibida. Por que querer que em um corpo tenha órgãos de nutrição, de locomoção, de movimento e para a reprodução? É preciso acabar com esta divisão através de um tipo de erotização geral de todo o corpo que seria inteiramente órgão de luxúria para um tipo de jubilação geral para aquele que se entregar à essa espécie de perversão, dando uma finalidade sexual ou afrodisíaca a todos elementos de seu corpo.\\
	\\
	O corpo sem órgãos...e a ideia do corpo sem órgãos vai levar a reflexões que, a meu ver, completamente interessantes. Eu creio que ele sabia - porque um dia pressionei-o contra à parede e interroguei-o sobre a questão - inspirado em um poeta patético que se pode gostar ou não, a saber: Antonin Artaud.\\
	\\
	As blasfêmias de Antonin Artaud contra Deus e a ordem divina tinham impressionado muito Deleuze. E havia na obra de Antonin Artaud, precisamente, diatribes contra o corpo. Antonin Artaud, no fim da sua vida, deteriorado pelo câncer... Antonin Artaud no qual nenhum órgão funciona corretamente...tudo é disfunção. Antonin Artaud vai se insurgir contra essa modelização do corpo à qual ele foi submetido em sua infância, e que lhe deu durante anos, assim ele diz, um corpo não-cristificado.\\
	\\
	Por que? Porque constituído segundo a Christi-formitas. Ora, é preciso libertar o corpo de toda Christi-formitas, dizia Artaud, e Deleuze vai retomar esta visão, é preciso libertar o corpo. É preciso "afrouxar" o corpo. Então, o que é "afrouxar" o corpo? O corpo sem órgãos.\\
	\\
	Eu cito:\\
	\\
	"Um corpo sem órgãos grita. Fizeram de mim um organismo. Forçaram-me indevidamente. Roubaram-me meu corpo. O julgamento de Deus o  arranca de sua imanência e faz dele um organismo, uma significação, um sujeito. Consideremos estes três estratos em sua relação conosco, isto é, aqueles que nos amarram mais diretamente: o organismo, o significante, a subjetivação. Tu serás um organismo. Tu articularás teu corpo senão tu não passarás de um depravado. Tu serás significante e significado, intérprete e interpretado senão tu não passarás de um desviado. Tu serás sujeito e fixado como tal. Sujeito denunciação rebatido por um sujeito denunciado senão tu não passarás de um vagabundo."\\
	\\
	As imprecações de Deleuze são realmente significativas. Fizeram dele... Precisamente porque ele recebeu uma educação cristã por parte de pais normais e sensatos. Fizeram dele, precisamente, o que ele recusou ser: um depravado, um desviante, um vagabundo. Pois, para ele, é são viver o corpo sem órgãos. Isto é, sem especificação, em cujo corpo todos os órgãos são polivalentes e podem preencher todas as funções que desejar fazer preenchê-las.\\
	\\
	Então, vejam! Não estamos longe da época moderna. Nada de novo foi inventado. A "queridinha" Najat Belkacem (ministra da educação no 2o. governo de Manuel Valls) não descobriu nada novo. E foi buscar na fonte, pois, de fato, foi essa gente lá que fez a escola e que se tornou maîtres à penser (mentores) da universidade. É claro... Era vosso servidor... (?) e mesmo contusões; e até braço quebrado (?) (Parece se referir a maio de 68 quando Pierre Magnard dava aulas em Dijon). Isso pelo fato dos mal feitos que vinham de Strasbourg até Dijon para pregar seu desgosto e golpear entre as cadeiras destruídas daquela pobre sala de aula que fora arrasada porque era a última escola onde se dedicava ainda a função de professores, apesar da greve que se tornaria geral. Eu ainda fui a única exceção a greve geral. Bom, isso não é um argumento filosófico. Mas temos outros argumentos.\\
	\\
	Então, eu continuo com o problema da desconstrução.\\
	\\
	Vocês viram um pouco o que representa nosso prezado Gilles Deleuze. Para ele, a multidão está no lugar da família.\\
	\\
	"O espaço sedentário é estriado por 'torres' [sic] (muros no texto original), cercas e caminhos entre cercas ao passo que o espaço nômade é liso (igual se fala de músculo liso e músculo estriado já que se trata, precisamente, de desorganizar o próprio organismo), somente marcado por traços que se apagam e se deslocam 'no panorama' (dans le sujet) [sic]."['com o trajeto' (avec le trajet)  no texto original e não 'dans le sujet']. Enquanto o migrante deixa um meio que se tornou amorfo e ingrato, o nômade é aquele que não pode e que não quer partir. Agarra-se a esse espaço liso onde a floresta recua, onde a estepe e o deserto crescem e inventa o nomadismo como resposta a esse desafio."\\
	\\
	É tudo o que nos oferece o autor de Mil Platôs. Transformar nossa amada França num deserto.\\
	\\
	Mas isso ele diz, ele trama e ele diz que é preciso fazer isso. É o preço a se pagar: essa grande revolução do ser humano. Então está aí o cerne dos desconstrutores. É ele, Gilles Deleuze.\\
	\\
	Mas eu gostaria de acrescentar ainda algumas palavras dele próprio. Pois é a ele que se deve a fórmula muito bonita de uma humanidade sem lugar de origem e sem história (hors-sol).\\
	\\
	"Humanidade sem lugar de origem como há endívias sem lugar de origem como há galinhas sem lugar de origem...é preciso uma humanidade sem lugar de origem porque não é necessário mais enraizamento."\\
	\\
	Pois há algo que ele teme absolutamente, a saber: a determinação pela terra natal (terroir). Ou seja, ser determinado por sua terra. A terra tem uma memória. A terra tem uma vontade. Ela não é apenas devaneio. A terra nos impõe sua lei. A terra é a única lei que devemos aceitar. Onde todos batalham.\\
	Então, ele quer recusar essa determinação pela terra. Essa determinação pelo enraizamento. Para ele, é preciso suprimi-la para todo sempre.\\
	\\
	Então, humanidade sem lugar de origem que não se lembra de suas proveniências. Sem lugar de origem quer dizer sem terra mas quer dizer, também, sem paternidade. Ele tem páginas terríveis sobre o que ele chama La Ceifal, que era uma revista que tinha sido criada por André Breton uns dez anos antes. \\
	\\
	La Ceifal. Cabeça cortada. Eu me lembro ainda da sua capa La Ceifal  - Cabeça cortada. Então, a cabeça cortada é a cabeça de Luís XVI, evidentemente. Era a cabeça do pai. Pois não há mais paternidade...cabeça cortada. E a La Ceifal queria que se produzisse uma humanidade que não requeresse mais a mínima filiação. Nossos Clowns de hoje, nossos legisladores atuais estão nessa mesma linha de pensamento. Eles não fazem nada mais do que disseminar aquilo que foi desenvolvido em um livro nos idos 1968 por Gilles Deleuze que foi meu desafiante desde os ano de 1947, 48, 49, 50.\\
	\\
	Eu entrei na Universidade em 1950. Eu o conheci somente em 1950 na universidade. Portanto, foi nessa época que ele já desenvolvia esse nobre pensamento. \\
	\\
	Portanto...avancemos na nossa apresentação dos desconstrutores.\\
	\\
	Michel Foucault\\
	\\
	O segundo será Michel Foucault. Ele é exatamente contemporâneo de Gilles Deleuze. Eles são da mesma idade. Ambos nasceram em 1925. Eles eram mais velhos do que eu dois anos. Michel Foucault, eu o conheci em Clermont-Ferrand. Eu era professor em khâgne em Clermont-Ferrand, Liceu Blaise Pascal e ele era professor da faculdade. Ele era um homem gentil, amigável. Permanecendo muito aceitável com quem não compartilha de seus hábitos.\\
	\\
	Eu não sentia nenhum constrangimento em ir ao bufê da estação de trem, tomar um copo de cerveja, na companhia de Foucault que me pedia para explicar-lhe que tinham sido os pensadores do XV e XVI séculos que ele incrivelmente ignorava.\\
	\\
	E Michel Foucault, portanto, encontrava-me todas as semanas quando ele voltava à Paris no bufê da estação de trem de Clermont-Ferrand para que eu lhe falasse sobre Raimondi, Colus (?) e de outros pensandores da renascença e da pré-renascença.\\
	\\
	E ele escrevia nessa época As palavras e as Coisas. E como ele podia fazer tudo - porque ele queria ir muito rápido. Ele me pegava, em alguns anos que lecionava no liceu, a quem ele pagava, para viajar à Paris e fosse até a biblioteca nacional recopiar as citações dos autores que eu citava abundantemente no meu curso. Pois ele não conhecia minhas anotações. Fiquei espantado quando As Palavras e as Coisas foi publicado de ver nos dois primeiros capítulos d'As Palavras e as Coisas uma quantidade de citações fora de contexto tomadas todas em sentido contrário porque ele não verificou em qual contexto elas estavam.\\
	\\
	Ele se contentava de recopiar as partes marcadas e puramente copiar e colar simplesmente. Mas não se deve espantar pois Michel Foucault vai a partir d'As Palavras e as Coisas [se tornar] um anti-intelectual(?). Mas Michel Foucault tinha uma particularidade, ele gostava muito de Deleuze e ele intencionava desenvolver o pensamento de Deleuze. E sua particularidade o tomava depois do segundo copo de cerveja. Ele se tornava um louco enfurecido quando ele falava do que lhe parecia dever ser eliminado para sempre da linguagem dos seres humanos: a palavra filiação.\\
	\\
	Não havia nada pior aos seus olhos do que ter um pai. Para ele, é o opróbrio por excelência. É preciso se desfazer da filiação. Libertar a humanidade é libertá-la da filiação. Não temos mais paternidade a partir de Foucault. Então não podemos nos espantar do que acontece, hoje em dia, com a GPA (Gestação para outrem, barriga de aluguel) e a PMA (Procriação médica assistida) estão em consonância com o que Foucault sempre ensinou.\\
	\\
	Foucault não cessava de dizer seu horror de que representava para ele o fato de ter nascido em Poitiers, vejam vocês, de um pai médico em Poitiers.\\
	\\
	"Ter tido um pai, ele me dizia, é o pior dos opróbrios."\\
	\\
	Um louco! [Mas foi] este louco que [se tornou o] maître à penser (intelectual orientador) para as gerações que seguiram. Pois, de fato, tudo que conhecemos, hoje, como crítica de filiação foi o que Foucault   preparou para esse mercado de ideias baratas. \\
	\\
	Então, eu não tenho tempo de seguir Foucault em todas suas elucubrações e eu gostaria de dizer algumas palavras sobre os outros dois desconstrutores.\\
	\\
	Pierre Bourdieu\\
	\\
	Pierre Bourdieu e Jacques Derrida. Então, Pierre Bourdieu é muito simples. É o mais simpático, o mais gentil do grupo. Ele foi para a Argélia prestar o serviço militar. Ele procurava o que poderia torná-lo diferenciado, o que poderia impulsioná-lo. E ele entra na residência geral (sede do representante do governo francês). Isto é, a direção fazendária e militar da Argélia.\\
	\\
	Ele foi à Argélia francesa com muita determinação. Depois ele volta e dois [anos] mais tarde parte novamente para a Argélia porque ele é convocado pelo comando militar ao qual pertencia e lá ele se torna membro do FLN (Frente de Libertação Nacional, partido político nacionalista argelino).\\
	\\
	Então, uma mutação brusca.\\
	\\
	Evidentemente, na biologia admite-se mutações bruscas mas, enfim, mesmo assim, é algo - e Jules Jean-Pierre me escuta - isso não impede de passar os atritos de um lado a outro. Ele me disse: \\
	\\
	"mas eu não era sensato porque eu era um filho do povo. Meu pai era jornaleiro e agricultor e é necessário que eu pense nas consequências e foi assim que eu passei para o lado do FLN."\\
	\\
	Então vai ser ele que vai escrever Os Herdeiros. Os Herdeiros, livro que, hoje, tratam como se fosse um tipo de bíblia no meio dos ditos "bem pensantes". O meio dos ditos "bem pensantes" fez de Os Herdeiros o livro imprescindível, indispensável. A tese de Os Herdeiros é muito simples: enquanto não se puder renunciar ao privilégio que a família pode dar na transmissão dos saberes não se terá uma sociedade igualitária.\\
	\\
	Portanto, se se vive numa sociedade verdadeiramente igualitária, e numa sociedade de homens livres, é preciso que não haja, em nenhum lugar, nenhuma família para a transmissão dos saberes.\\
	\\
	Este pensamento vai se tornar um inimigo íntimo para mim porque eu me fizera, já na época, o campeão da transmissão e ele era realmente o adversário da transmissão. Nós nos opúnhamos frequentemente sobre o tema da transmissão. E eu chegarei ao quarto desconstrutor que é Jacques Derrida.\\
	\\
	Jacques Derrida\\
	\\
	Então, a carreira de Derrida vocês a conhecem. Jacques Derrida estava à khâgne (classe preparatória) comigo no liceu Louis-le-Grand no último ano. Ele estava lá, como Bourdieu e eu, no mesmo ano. Em 1950. E Jacques Derrida indicava ser um fervoroso proponente da desconstrução custasse o que custasse. Foi assim que ele conheceu alguns descontentes quando ele quis se candidatar à universidade.\\
	\\
	Eu era do conselho consultivo que selecionava os professores universitários e, por duas vezes, fui levado à votar para que ele não fosse selecionado. O que não me impedia de ter boas relações com ele porque eu me expliquei por telefone. Eu lhe disse que se talvez cessasse de ser como era, gostaria de aprová-lo. Mas que aquela pose, aquela impostura que ele mantinha para um posto filosófico era algo absolutamente detestável. Mas, todavia, ele ia conhecer a fortuna. Foi em 1981.\\
	\\
	A fortuna ele a conheceu por sua prisão por tráfico de drogas na fronteira ao retornar de uma viagem da Europa Central. Mala com fundo falso, drogas pesadas...ele vivia disso já que ele não tinha posto na universidade e ele não queria dar aulas no liceu. E ele sucumbiria a ação da justiça se uma "divina" surpresa não tivesse intervindo em sua carreira: a eleição "miraculosa" de um certo François Mitterrand. Do qual ele se tornará, de um dia para o outro, membro do conselho filosófico e que, por isto, será ilibado de seus crimes.\\
	\\
	Mitterrand funda o Colégio Internacional de Filosofia para Derrida. E Derrida, a partir daí, com a situação tranquila frente a justiça, será até mesmo colocado numa situação de superioridade para julgar seus colegas de universidade que ele voltará a encontrar. Portanto, a fortuna de Derrida é menos suas obras que são todas ineptas - uns mais que outras - e rudemente mal redigidas do que a "graça" presidencial que fez dele o pensador (filósofo) oficial.\\
	\\
	Então o que ele acrescentou a desconstrução?\\
	\\
	Nada. A não ser que ele a generaliza. Ele a praticou sistematicamente e em toda parte. Quer dizer que todos os binômios constituídos de um tipo, de uma maneira canônica devem ser destruídos. Deve-se colocar por desígnio que o mal é belo e que o bem deve-se procurar precisamente nos antípodas dos locais onde se procura habitualmente. E, então, seu desafio será de desenvolver isso em um plano de um tipo de erotização geral da palavra escrita e da ação que lhe permitirá de justificar essa desconstrução generalizada.\\
	\\
	Então está aí o que eu tinha de dizer sobre os quatro cavaleiros do apocalipse.\\
	\\
	O Desastre\\
	\\
	O que eu gostaria é de tentar tirar uma sombra, mais o sonho do que eu lhes apresentei sobre o modo conflitual e, deliberadamente, opositivo. Eu o farei por uma ficção que não é imaginária, já que é a relação de uma lembrança muito precisa. Vocês poderiam me perguntar se, fora da camaradagem ocasional que eu tive com esses quatro personagens, tive a possibilidade de encontrá-los todos juntos e debater com eles. Isto aconteceu em Saint-Germain-des-Prés na casa de Marguerite Duras que recebia Maurice Blanchot.\\
	\\
	Este nome não lhes diz, talvez, mais nada. Ele foi um pensador muito sensato, muito distinto e muito sutil. Mas talvez, hoje, completamente esquecido. E, no entanto, ele é, talvez, aquele que sem querer foi o que mais me ajudou no meu combate. \\
	\\
	Maurice Blanchot era um crítico literário muito mais velho do que nós, já que ele tinha nascido, se eu não me engano, em 1907 - ele era vinte anos mais velho do que eu. E ele reunia o que eles acreditavam ser as cabeças pensantes da universidade da época, de maneira que eu tive a honra de reunir na casa de Marguerite Duras, que nos ofereceu seu apartamento, com Bourdieu, com Derrida, com Deleuze e com Foucault.\\
	\\
	O que se destacou de suas reflexões?\\
	\\
	O que eu guardei foi o silêncio de Blanchot que nunca tomava partido no debate. E que tinha a honestidade de impor minha presença enquanto eu estava dissonante e mal visto naquele meio. É preciso dizer que Maurice Blanchot fazia a grande diferença. Ele fora discípulo de Maurras. E foi com este "título" que ele tinha por mim alguma simpatia, digamos assim, ou alguma indulgência. E ele tinha durante a guerra colaborado com os comunistas, não por marxismo, mas porque ele tinha ajudado a passar judeus para a Suíça. E como ele era de um grande generosidade, sua casa era um pouco uma casa de acolhimento.\\
	\\
	Ora, o que eu guardo de Blanchot?\\
	\\
	Toda uma literatura que aparecerá bem depois e que terá por tema fundamental o desastre. E se você quiser conhecer algo de Blanchot, hoje, compre A Escritura do Desastre.\\
	\\
	O que é o desastre?\\
	\\
	Quando Blanchot nos recebe entre 1947 e 1952, vivíamos naquele período em que a ameaça comunista era grande sobre a França. Período em que tememos, precisamente, a tomada do poder pelo partido - lembrem-se do próprio ano de 1947! E nosso admirável Blanchot num tom monocórdico com uma atonia absolutamente surpreendente, escritor de uma escrita que eu qualifico de uma escrita sóbria - pois ele não manifesta nenhuma paixão -, nos explica o que é o desastre.\\
	\\
	Ele nos dizia para permanecermos calmo: \\
	\\
	"não vós impressionai! Vós viveis um desastre, quer dizer que o homem da vossa geração perdeu seu astro."\\
	\\
	O desastre é isto.\\
	\\
	"A marcha para a estrela chegou ao fim. O lavrador de outrora caminhava em direção à estrela para ir ao seu primeiro campo arado. Hoje não há mais caminho para a estrela. O céu se escureceu, o céu ficou turvo. Não há mais sentido para buscar ou algo que seja. É o desastre."\\
	\\
	Mas, então, o que fazer nesse desastre?\\
	\\
	\\
	Mas no desastre é bom aceitar prudentemente que nada mais manifesta um sentido. Nem sobre a terra nem no céu já que as constelações, elas mesmas, parecem  estar apagadas. Quanto ao perfume das flores e as cores da folhagem elas não lhes diz mais nada, pois o sentido desapareceu. O sentido do sentido - a capacidade de sentir está, de alguma maneira, extinta em nossa humanidade. Por isso, nós vivemos o desastre. Mas o desastre jamais foi tão certo do que isso.\\
	\\
	Por quê?\\
	\\
	Acredita-se que a casa se desmoronou. Acredita-se que ela já pereceu. Todos as outras desapareceram. Elas são fantasmagóricas. E, no entanto, você, fiel, último rebento com atributos, você bate a porta e você vê, você se coloca nos velhos muros... nos velhos muros que você não sabe absolutamente ao certo se eles ainda existem, ou se eles são puramente e simplesmente ilusórios e nos dizia Blanchot: \\
	\\
	"vós entrais no reino da precariedade."\\
	\\
	Ora, o que é o reino da precariedade?\\
	\\
	É aquele cujos edifícios, as fundações das construções não aguentam mais suas vigas mas, simplesmente, em virtude de nossa oração, nosso pedido. Précarité, pré carré, prier [Precariedade, o que é só da nossa alçada, rezar (há uma aliteração e parece haver um jogo de palavras].\\
	\\
	Ele nos dizia: \\
	\\
	"Então, é o último recurso que vos resta. Não pergunteis vós se a casa está ainda lá, ide servi-la como se ela jamais estivesse derrubada. Ela se mantém apenas por vossa oração."\\
	\\
	Aí está a lição que nos deu Maurice Blanchot, naqueles anos, diante de nossos quatro desconstrutores que não tinham nenhuma consideração por isso.\\
	\\
	Eu jamais esquecerei o olhar terno de Blanchot sobre seu empregado quando ele lhe dizia:\\
	\\
	"Não deixe, meu caro amigo, de rezar." \\
	\\
	Obrigado.
\end{document}